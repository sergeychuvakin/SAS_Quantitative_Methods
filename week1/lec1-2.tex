
\documentclass[t, 11pt]{beamer}
\pdfmapfile{+sansmathaccent.map}
%%% Работа с русским языком
\usepackage{cmap}				
\usepackage{mathtext} 				
\usepackage[T2A]{fontenc}		
\usepackage[utf8]{inputenc}			
\usepackage[russian, english]{babel}	

\usetheme{Montpellier}
\usecolortheme{beaver} % Цветовая схема



%%% Работа с картинками
\usepackage{graphicx}

\usepackage{csquotes}

\hypersetup{				
	colorlinks=true,       	
	linkcolor=blue,          
	citecolor=black,       
	filecolor=magenta,      
	urlcolor=red           
}
%% табличка
\usepackage{booktabs, caption, makecell}
\usepackage{threeparttable}


\title{Intro to Quantitative methods}
\subtitle{}
\author{Chuvakin Sergey}
\date{\today}
\institute[<<Anthropology>>]{<<School of Advanced Studies>>}

\begin{document}
	
	\frame[plain]{\titlepage}
	\section{lection 1 Intro to Course}
	
	
	\subsection{Welcome}
	\begin{frame} \label {welcome}
		\frametitle{\insertsection} 
		\frametitle{\insertsubsection} 
		\begin{itemize}
			\item \textbf{About me:} Data Scientist in Arcadia inc. (SPb), mostly worked with pharma companies. Former sociologist with focus on criminology. 
			\item \textbf{My interests:} Business and Research Analytics, Natural Language Processing, Python, Digital Humanities, Robotics
			\item \textbf{Feel free to contact me:} \begin{itemize}
				\item Email: sergeychuvakin1@mail.ru
				\item telegram: @sergechuvakin
			\end{itemize}
			\item \textbf{Office hours:} TBA
		\end{itemize}
	\end{frame}
	
	\subsection{Quantitative methods}
	\begin{frame}\label{}
		\frametitle{\insertsection}
		\frametitle{\insertsubsection}
		
		\textbf{Quantitative methods} -- is a set of rules and algorithms to reach stable, sustained results of research. 
		
		\textbf{Research} -- common term for defining procedure of finding answer on question.
		
	\end{frame}
	
	
	
	\begin{frame}\label{}
		\frametitle{\insertsection}
		\frametitle{\insertsubsection}
		
		After completing this course, you will be able: 
		\begin{itemize}
			\item to read and understand (sic!) quantitative research papers 
			\item to speak the language of data fluently
			\item move further to Data Science and Machine Learning 
			\item able to understand and explain to others such words as <<variable>>, <<distribution>>, <<regression>>,<<p-value>>, etc.
			\item able to choose statistical methods appropriate to your research problem 
			\item use R for your programming needs
		\end{itemize}
	\end{frame}
	
	
	\subsection{Course Structure} \label{}
	\begin{frame}
		\frametitle{\insertsection}
		\frametitle{\insertsubsection}  
		\begin{itemize}
			\item Week 1 - Intro $\xi$
			\item Week 2 - R Language $\xi$
			\item Week 3 - Key statistical concepts $\xi$
			\item Week 4 - Data Management $\xi$
			\item Week 5 - Basic statistical tests $\phi$
			\item Week 6 - Regression $\xi$ 
			\item Week 7 - Regression advanced
			\item Week 8 - Methodology of Quantitative research $\xi$ $\phi$
		\end{itemize}
		\vspace{1cm}
		$\xi$ -- Quiz on the week
		
		$\phi$ -- Home Assignment
	\end{frame}
	
	\begin{frame}
		\frametitle{\insertsection}
		\frametitle{Prerequisites}  
		\begin{itemize}
			
			\item Math (at least school level)
			\item Calculus (basics) (recap today)
			\item Linear Algebra (basics) (recap today)
			\item Computer literacy 
			
		\end{itemize}
	\end{frame}
	
	
	\subsection{Grading}
	
	\begin{frame}\label{}
		\frametitle{\insertsection}
		\frametitle{Major  (approximate)}
		\begin{table}[]
			\begin{threeparttable}
				\begin{tabular}{@{}lll@{}}
					\toprule
					\textbf{Assignment or Task}   & \textbf{Due date/s} & \textbf{Percent} \\ \midrule
					Recap of math and Probability & 22 November         & 8       \\
					R language practice           & 29 November         & 8      \\
					Data management practice      & 13 December         & 8                \\
					Regression practice           & 27 December         & 8                \\
					1 Home assignment             & 11 January          & 30               \\
					Design research               & 20 January          & 8                \\
					2 Home assignment             & 20 January          & 30               \\ \bottomrule
				\end{tabular}
				\begin{tablenotes}\footnotesize
					\item[*]  it's not the end version, small changes can be (TBA)
				\end{tablenotes}
			\end{threeparttable}
		\end{table}
	\end{frame}
	
	
	\begin{frame}\label{}
		\frametitle{\insertsection}
		\frametitle{Minor  (approximate)}
		\begin{table}[]
			\begin{threeparttable}
				\begin{tabular}{@{}lll@{}}
					\toprule
					\textbf{Assignment or Task}   & \textbf{Due date/s} & \textbf{Percent} \\ \midrule
					Recap of math and Probability & 22 November         & 15      \\
					R language practice           & 29 November         & 15      \\
					Data management practice      & 13 December         & 15                \\
					Regression practice           & 27 December         & 15                \\
					1 Home assignment             & 11 January          & 25               \\
					Design research               & 20 January          & 15                \\
				\end{tabular}
				\begin{tablenotes}\footnotesize
					\item[*]  it's not the end version, small changes can be (TBA)
				\end{tablenotes}
			\end{threeparttable}
		\end{table}
	\end{frame}
	
	
	
	\subsection{Reading}
	\begin{frame}\label{}
		\frametitle{\insertsection}
		\frametitle{\insertsubsection}
		\begin{itemize}
			\item Field A., J. Miles, and Z. Field. 2012.Discovering Statistics Using R. SAGE publications ltd
			\item Wickham, H., and Grolemund, G. 2016.R for data science.O’Reilly Media
		\end{itemize}
	\end{frame}
	
	\subsection{Software}
	\begin{frame}\label{}
		\frametitle{\insertsection}
		\frametitle{\insertsubsection}
		\begin{itemize}
			\item \href{https://www.r-project.org/}{R} - Language itself
			\item \href{https://www.rstudio.com/products/rstudio/}{Rstudio} - Application for comfortable work
			\item \href{https://github.com/sergeychuvakin/SAS_Quantitative_Methods/}{GitHub} - all the materials stored here 
		\end{itemize}
	\end{frame}
	
	
	\subsection{Intro - today}
	\begin{frame}\label{}
		\frametitle{\insertsection}
		\frametitle{\insertsubsection}
		
		\begin{itemize}
			\item Course introduction 
			\item Calculus and linear algebra recap
			\item quiz 
		\end{itemize}
	\end{frame}	
	
	\subsection{R Language}
	\begin{frame}\label{}
		\frametitle{\insertsection}
		\frametitle{\insertsubsection}
		
		\textbf{R} -- general purpose programming language. We will practice:
		\begin{itemize}
			\item writing a good readable code
			\item use basic built-in functions 
			\item modify  and create data tables
			\item prepare data to analysis
			\item use statistical functions
			\item run models and diagnostics 
			\item visualize data and models
			\item use RMarkdown extension
			\item a lot of other activities 
		\end{itemize}	  
	\end{frame}	
	
	\subsection{Key statistical concepts}
	\begin{frame}\label{}
		\frametitle{\insertsection}
		\frametitle{\insertsubsection}
		
		\begin{itemize}
			\item what is variable
			\item types of variables
			\item sample and population
			\item Representativeness 
			\item central measures
			\item distributions 
			\item standard deviation
		\end{itemize}	  
	\end{frame}	
	
	
	\subsection{Data management and visualization}
	\begin{frame}\label{}
		\frametitle{\insertsection}
		\frametitle{\insertsubsection}
		
		\begin{itemize}
			\item Relational data structure
			\item What is dplyr
			\item What is ggplot2
			\item basic data transformations
			\item visualization basics
		\end{itemize}	  
	\end{frame}		
	
	\subsection{Basic statistical tests}
	\begin{frame}\label{}
		\frametitle{\insertsection}
		\frametitle{\insertsubsection}
		
		\begin{itemize}
			\item compare two means 
			\item compare more than two means
			\item t-test
			\item chi-square
			\item significance
			\item ANOVA
			\item non-parametrics tests
			\item correlation 
		\end{itemize}	  
	\end{frame}			
	
	\subsection{Regression}
	\begin{frame}\label{}
		\frametitle{\insertsection}
		\frametitle{\insertsubsection}
		
		\begin{itemize}
			\item assumptions 
			\item formula 
			\item t-test
			\item interpretation
			\item significance
		\end{itemize}	  
	\end{frame}			
	
	\subsection{Regression advanced}
	\begin{frame}\label{}
		\frametitle{\insertsection}
		\frametitle{\insertsubsection}
		
		\begin{itemize}
			\item Diagnostics 
			\item types of errors 
			\item GLM
		\end{itemize}	  
	\end{frame}			
	
	\subsection{Methodology}
	\begin{frame}\label{}
		\frametitle{\insertsection}
		\frametitle{\insertsubsection}
		
		\begin{itemize}
			\item Design  of research 
			\item \textbf{A\slash B} tests
		\end{itemize}	  
		
	\end{frame}			
	
	
\end{document}
