
\documentclass[t, 11pt]{beamer}
\pdfmapfile{+sansmathaccent.map}
%%% Работа с русским языком
\usepackage{cmap}				
\usepackage{mathtext} 				
\usepackage[T2A]{fontenc}		
\usepackage[utf8]{inputenc}			
\usepackage[russian, english]{babel}	

\usetheme{Montpellier}
\usecolortheme{beaver} % Цветовая схема



%%% Работа с картинками
\usepackage{graphicx}

\usepackage{csquotes}

\hypersetup{				
	colorlinks=true,       	
	linkcolor=blue,          
	citecolor=black,       
	filecolor=magenta,      
	urlcolor=red           
}
%% табличка
\usepackage{booktabs, caption, makecell}
\usepackage{threeparttable}

%% график нормального распределения 
\usepackage{tikz}
\usepackage{xcolor}
\usepackage{pgfplots}
\pgfplotsset{compat=1.7}

%% доп символы
\usepackage{newunicodechar}

\newcommand\Warning{%
	\makebox[1.4em][c]{%
		\makebox[0pt][c]{\raisebox{.1em}{\small!}}%
		\makebox[0pt][c]{\color{red}\Large$\bigtriangleup$}}}%

\newunicodechar{⚠}{\Warning}

\title {Key statistical concepts}
\subtitle{}
\author{Chuvakin Sergey}
\date{\today}
\institute[<<Anthropology>>]{<<School of Advanced Studies>>}

\begin{document}

	
	\frame[plain]{\titlepage}		
	
	\section{Outline}
	
		\begin{frame} 
			\frametitle{\insertsection} 
			\begin{itemize}
				\item Random variable
				\item Distribution
				\item Distribution types
				\item Central measures
				\item Dispersion 
				\item Standart deviation 
				\item Types of variables 
				\item Population
				\item Sample
				\item law of large numbers (maybe)
				\item central limit theorem (maybe)
			\end{itemize}
		\end{frame}
	
	\section{Random Variable}
		\subsection{Definition}

	\begin{frame} 
		\frametitle{\insertsection} 
		\framesubtitle{\insertsubsection}
	Variable - \textbf{\emph{varying}} values. 
		
	\textbf{RV} - aka  random quantity, aleatory variable, or stochastic variable - is a variable whose value is unknown or a function that assigns values to each of an experiment's outcomes.
	
	\vspace{1cm}
	
	\emph{Examples:} - tips for waiter, number of people in a line, number of insects under a bed and a lot of other examples. The idea that it's unlimited number. Everything potetntilaly could be a random variable. 
	\end{frame}
	
	
	\section{Distribution}
	\subsection{Definition}
		\begin{frame} 
		\frametitle{\insertsection} 
		\framesubtitle{\insertsubsection}
		In statistics, a probability distribution is a mathematical description of a random variable in terms of the probabilities of it's particular possible values.
		
		\vspace{1cm}
		
		 Put it simpler - distribution - possible values of random variable. 
		 
		 \vspace{1cm}
		 
		NB: It's a function, therefore there are input and output.
	\end{frame}
	
	\subsection{Types}
	\begin{frame} 
		\frametitle{\insertsection} 
		\framesubtitle{\insertsubsection}
		
		\href{https://seeing-theory.brown.edu/probability-distributions}{Tap here!} - interactive types of various distributions! 
		
		\vspace{1cm}
		
		Let us see it together! 
		
		What worth noticing: 
		\begin{enumerate}
		\item it can be descrite and continious
		\item each has extra parameters 
		\item shape varies
		\item some combinations of parameters of different distributions looks alike!  
	 \end{enumerate}
	\end{frame}

	\begin{frame} 
	\frametitle{\insertsection} 
	\framesubtitle{\insertsubsection}
	
	Exercise - try to guess the following! 
	
	\begin{enumerate}
		\item meaning of (at least some of) parameters
		\item difference between descrete and continious 
		\item what is probability mass (density) function (PDF) $f(x)$
		\item what if cumulative distribution function (CDF) $F(x)$
	\end{enumerate}

	\vspace{1cm}
	
	Do not be upset if not all of above is clear!  The more important is to grasp intuition...

\end{frame}


	\pgfmathdeclarefunction{gauss}{2}{\pgfmathparse{1/(#2*sqrt(2*pi))*exp(-((x-#1)^2)/(2*#2^2))}}
	
	\section{Normal distribution}
	\subsection{Plot}
	\begin{frame} 
		\frametitle{\insertsection} 
		\framesubtitle{\insertsubsection}
		\begin{center}
		\begin{tikzpicture}
			
			\begin{axis}[no markers, domain=0:10, samples=100,
				axis lines*=left, xlabel=$X$, ylabel=axis $y$,
				height=6cm, width=10cm,
				xticklabels={Test A,$99\%$,$95\%$,$68\%$, $\mu$, $\sigma$,$2\sigma$,$3\sigma$}, ytick=\empty,
				enlargelimits=false, clip=false, axis on top,
				grid = major]
				\addplot [fill=cyan!20, draw=none, domain=-3:3] {gauss(0,1)} \closedcycle;
				\addplot [fill=orange!20, draw=none, domain=-3:-2] {gauss(0,1)} \closedcycle;
				\addplot [fill=orange!20, draw=none, domain=2:3] {gauss(0,1)} \closedcycle;
				\addplot [fill=blue!20, draw=none, domain=-2:-1] {gauss(0,1)} \closedcycle;
				\addplot [fill=blue!20, draw=none, domain=1:2] {gauss(0,1)} \closedcycle;
			\end{axis}
		\end{tikzpicture}
		
	\end{center}
	
	\end{frame}

	\subsection{Fomula}
	\begin{frame} 
		\frametitle{\insertsection} 
		\framesubtitle{\insertsubsection}
		$$P(x) = \frac{1}{{\sigma \sqrt {2\pi } }}e^{{{ - \left( {x - \mu } \right)^2 } \mathord{\left/ {\vphantom {{ - \left( {x - \mu } \right)^2 } {2\sigma ^2 }}} \right. \kern-\nulldelimiterspace} {2\sigma ^2 }}}$$
		\begin{itemize}
			\item $P(x)$ aka $y$
			\item $\sigma$ - standart deviation
			\item $\mu$ - mean aka Expected value $E(x)$
			\item $e$ - e number (2.7\texttildelow)
			\item $\pi$ Pi number (3.14\texttildelow)
			\item $x$ - value from random variable
			\end{itemize}
		
	\vspace{0.5cm}
	
	\emph{Intuition behind - probability of $x$ assumed it distributed normally.} 
	\end{frame}

	\section{Back to variables}
	\subsection{Examples in sociology and anthropology}
	\begin{frame} 
		\frametitle{\insertsection} 
		\framesubtitle{\insertsubsection}
	RV - frequently used as quality (or feature) of person or some phenomena. 
	
	Examples of pseudo normal distribution:
	
	\begin{itemize}
		\item Age 
		\item Heigh
		\item Salary 
		\item Number of robberies in a country 
		\item Number of votes  during elections
		\item Number of cigarettes smoked
		%item Favorite color 
		%\item Gender
		%\item Race
	\end{itemize}
	
	
	\end{frame}
	
\section{Formal Statistics}
\subsection{central tendency}
\begin{frame} 
	\frametitle{\insertsection} 
	\framesubtitle{\insertsubsection}
	
	\begin{itemize}
		\item Mean
		\item Mode
		\item Median
	\end{itemize}

%U+2757
\Warning  In trully normal distrubution Mean  $\simeq$  Mode $\simeq$  Median
	
\end{frame}	

\begin{frame} 
	\frametitle{\insertsection} 
	\framesubtitle{\insertsubsection}
	The mean of a distribution is the arithmetic mean, or the <<average>>
	
	$${\displaystyle A={\frac {1}{n}}\sum _{i=1}^{n}a_{i}={\frac {a_{1}+a_{2}+\cdots +a_{n}}{n}}}$$
	
	$$\mu(X) = \frac{\sum_{i=1}^{n}x^{i}}{n}$$


	
\end{frame}	

\begin{frame} 
	\frametitle{\insertsection} 
	\framesubtitle{\insertsubsection}
	
	The median is the value separating the higher half of sample,a population, or a probability distribution, from the lower half.
	\begin{enumerate}
	\item The median is the “middle” value of a [ordered] data set.
	\item Let there be a variable v1: 20,7,23,17,21,5,19,3,11 
	\item To compute the median of v1, sort the variable into ascending order: 3,5,7,11,17,19,20,21,23
	\item Pick one in center (17)
	\end{enumerate}
	
	\end{frame}	

\begin{frame} 
	\frametitle{\insertsection} 
	\framesubtitle{\insertsubsection}
	
	The mode is the value that appears most often in a set of data
	\begin{enumerate}
	\item Consider data 2,5,7,6,7,9,2,0,5,3,3,7,7,8.
	\item The mode is 7
	\item What about 3,7,4,2,3,1,0,7,9,6,3,7,4,11?
	\item The mode of a continuous probability distribution is the value x at which its probability density function has its maximum value
	\item The mode is at the peak of the distribution
	\end{enumerate}
	
\end{frame}	

\pgfmathdeclarefunction{gauss1}{3}{\pgfmathparse{1/(#3*sqrt(2*pi))*exp(-((#1-#2)^2)/(2*#3^2))}}

\begin{frame} 
	\frametitle{\insertsection} 
	\framesubtitle{\insertsubsection}
	\begin{center}
\begin{tikzpicture}
	
	\begin{axis}[
		no markers
		, domain=-7.5:25.5
		, samples=100
		, ymin=0
		, axis lines*=left
		, xlabel=
		, every axis y label/.style={at=(current axis.above origin),anchor=south}
		, every axis x label/.style={at=(current axis.right of origin),anchor=west}
		, height=5cm
		, width=13cm
		, xtick=\empty
		, ytick=\empty
		, enlargelimits=false
		, clip=false
		, axis on top
		, grid = major
		, hide y axis
		, hide x axis
		]
		
		
		
		%\draw [help lines] (axis cs:-3.5, -0.4) grid (axis cs:3.5, 0.5);
		
		% Normal Distribution 1
		\addplot[blue, ultra thick,restrict x to domain=-6:6] {gauss1(x, 0, 1.75)};
		\pgfmathsetmacro\valueA{gauss1(0, 0, 1.75)}
		\draw [dashed, thick, blue] (axis cs:0, 0) -- (axis cs:0, \valueA);
		\node[below] at (axis cs:0, -0.02)  {\Large \textcolor{blue}{$\mu_{1}$}};
		\draw[thick, blue] (axis cs:-0.0, -0.01) -- (axis cs:0.0, 0.01);
		
		% Normal Distribution 2
		\addplot[green, ultra thick,restrict x to domain=3:15] {gauss1(x, 9, 1.75)};
		\draw [dashed, thick, green] (axis cs:9, 0) -- (axis cs:9, \valueA);
		\node[below] at (axis cs:9, -0.02)  {\Large \textcolor{green}{$\mu_{2}$}};
		\draw[thick, green] (axis cs:9, -0.01) -- (axis cs:9, 0.01);
		
		% Normal Distribution 3
		\addplot[red, ultra thick,restrict x to domain=12:24] {gauss1(x, 18, 1.75)};
		\draw [dashed, thick, red] (axis cs:18, 0) -- (axis cs:18, \valueA);
		\node[below] at (axis cs:18, -0.02)  {\Large \textcolor{red}{$\mu_{3}$}};
		\draw[thick, red] (axis cs:18, -0.01) -- (axis cs:18, 0.01);
		
	\end{axis}
	
	
\end{tikzpicture}
	\end{center}
\end{frame}	


\begin{frame} 
	\frametitle{\insertsection} 
	\framesubtitle{\insertsubsection}
	
\Warning 1 Guess why median is more representative one?

\vspace{1cm}

\Warning 2 What is the synonym for representative? 
	
\end{frame}	

\section{Dispersion and std}
\subsection{Formula}
\begin{frame} 
	\frametitle{\insertsection} 
	\framesubtitle{\insertsubsection}
Range is the difference between the smallest and the largest observation in the sample.	

\begin{itemize}
	\item Variance is the average of all squared deviations from the mean:
	$$\sigma^2 = \frac{\sum^{n}_{i=1} (x_i - \mu(X))^2}{n}$$
 	\item The larger this value, the greater the dispersion of the observations around the mean value, the more heterogeneous sample (the less informative mean).
 	\item The standard deviation (denoted as $\sigma$ or $s$) is the square root of the variance
\end{itemize}
\end{frame}	


\section{Variable types}
\subsection{list}

\begin{frame} 
	\frametitle{\insertsection} 
	\framesubtitle{\insertsubsection}

	\begin{enumerate}
		\item \tikz\draw[red,fill=red] (0,0) circle (.5ex); continious $(-\infty:+\infty)$
		\item \tikz\draw[blue,fill=blue] (0,0) circle (.5ex); nominal (colors)
		\item \tikz\draw[green,fill=green] (0,0) circle (.5ex); ordered (ranks in the army)
	\end{enumerate}

	
\end{frame}

\subsection{ Examples}

\begin{frame} 
	\frametitle{\insertsection} 
	\framesubtitle{\insertsubsection}
	
	\tikz\draw[red,fill=red] (0,0) circle (.5ex); Continious:
	\begin{enumerate}
		\item Age of person
		\item Salaries in a company
	\end{enumerate}
	\tikz\draw[blue,fill=blue] (0,0) circle (.5ex); Nominal:
	\begin{enumerate}
		\item Religion of person
		\item Prefered candidate at elections
		\item Gender
	\end{enumerate}

	\tikz\draw[green,fill=green] (0,0) circle (.5ex); Ordered: 
	\begin{enumerate}
		\item Number of smoked sigaretes 
		\item Number of children in a family 
	\end{enumerate}
	
\end{frame}

\subsection{Notions}

\begin{frame} 
	\frametitle{\insertsection} 
	\framesubtitle{\insertsubsection}
	
	\tikz\draw[red,fill=red] (0,0) circle (.5ex); Continious - takes any real number - it can be normally distributed or somehow else. 
	
	\vspace{1cm}
	
	\tikz\draw[blue,fill=blue] (0,0) circle (.5ex); Nominal - always descrete. The only thing can be done upon this - count unique values.  The values \emph{can not be} compared!  

	\vspace{1cm}
	
	\tikz\draw[green,fill=green] (0,0) circle (.5ex); Ordered - unlike nominal scal it can be compared, but we do not know the granularity of difference. Descrete as well!

\end{frame}


\section{Population and  Sample}
\subsection{Formula}
\begin{frame} 
	\frametitle{\insertsection} 
	\framesubtitle{\insertsubsection}
\textbf{Popuation} - theoretical measure of your object of research.  

\vspace{1cm}

\textbf{Sample} - Real number of observations

\vspace{1cm}

\textbf{Representetivness} - How sample reflects quality of population 

\end{frame}	

\section{Important Laws}

\subsection{Law of large numbers}
\begin{frame} 
	\frametitle{\insertsection} 
	\framesubtitle{\insertsubsection}
	
	The law of large numbers, in probability and statistics, states that as a sample size grows, its mean gets closer to the average of the whole population
	
\end{frame}	



\subsection{Central Limit Theorem}
\begin{frame} 
	\frametitle{\insertsection} 
	\framesubtitle{\insertsubsection}
	
	The central limit theorem states that if you have a population with mean $\mu$ and standard deviation $\sigma$and take sufficiently large random samples from the population with replacement, then the distribution of the sample means will be approximately normally distribute
	
\end{frame}	




%TODO: sample, population, types of variables 

\end{document}
	